\documentclass[
a4paper,					% alle weiteren Papierformat einstellbar
%landscape,					% Querformat
11pt,						% Schriftgr�ߟe (12pt, 11pt (Standard))
BCOR0.8cm,					% Bindekorrektur, bspw. 1 cm
%DIVcalc,					% f�hrt die Satzspiegelberechnung neu aus
twoside,					% einseitiges Layout
%twocolumn,					% zweispaltiger Satz
%openany,					% Kapitel k�nnen auch auf linken Seiten beginnen
%halfparskip*,				% Absatzformatierung s. scrguide 3.1
%notitlepage,				% in-page-Titel, keine eigene Titelseite
%chapterprefix,				% vor Kapitel�berschrift wird "Kapitel Nummer" gesetzt
%appendixprefix,			% Anhang wird "Anhang" vor die ܜberschrift gesetzt 
%normalheadings,			% ܜberschriften etwas kleiner (smallheadings)
%idxtotoc,					% Index im Inhaltsverzeichnis
%liststotoc,				% Abb.- und Tab.verzeichnis im Inhalt
%bibtotoc,					% Literaturverzeichnis im Inhalt
bibliography=totoc,
%leqno,						% Nummerierung von Gleichungen links
%fleqn,						% Ausgabe von Gleichungen linksb�ndig
%draft						% �berlangen Zeilen in Ausgabe gekennzeichnet
]{scrreprt}

% ============================ Packages ============================ 

\usepackage[inner=3.0cm,outer=2.5cm,top=1.5cm,bottom=1.5cm,includeheadfoot]{geometry}
									% Einstellungen der Seitenr�nder

\usepackage[T1]{fontenc}				% neue Rechtschreibung
\usepackage[latin1]{inputenc}		         % Umlaute erm�glichen
\usepackage[english, german]{babel}
\usepackage{fancyhdr}				% f�r Kopf- und Fuߟzeile
\usepackage[hyperref]{xcolor}		
\definecolor{MyBlue}{RGB}{50,79,132}		% f�r Kopf- und Fu�zeile
\usepackage[colorlinks, linktocpage, linkcolor=MyBlue]{hyperref}
\usepackage{color}					% Farbe
\usepackage[pdftex]{graphicx}
\usepackage{float}
\usepackage{listings}				% Programm-Code
\usepackage{longtable}				% Tabellen �ber mehrere Seiten
\usepackage{siunitx}				% Einheiten nach SI-Norm
\usepackage{listings}                                    % Verwendung von Matlab-Code
\usepackage{amsmath}
\usepackage{amssymb}
\usepackage[tight]{subfigure}
\usepackage{empheq}
\usepackage{overcite}
\usepackage{url}
\usepackage{framed}
\usepackage{pdfpages}
\usepackage[compact]{titlesec}
% ============================ Kopf- und Fu�zeile ==========================

\pagestyle{fancy}
\fancyhf{}

\fancyhead[CO,CE]{\scshape\leftmark}	% Kopfzeile links [L], mitte [C], rechts [R]
\fancyfoot[RO,LE]{\thepage}				% Fu�zeile
\renewcommand{\headrulewidth}{0.5pt}	% Linie oben
\renewcommand{\footrulewidth}{0pt}		% Linie unten
\pagenumbering{roman}

%============================ new Definitions ============================
%============================ Befehlsdefinitionen============================

\newcommand{\beq}{\begin{eqnarray}}		% Abk�rzung f�r nummerierte Gleichungen
\newcommand{\eeq}{\end{eqnarray}}		% Ab�rzung f�r nummerierte Gleichungen
\newcommand{\nn}{\nonumber}				% Abk�rzung f�r nummernlose Gleichung
\newcommand{\dm}{\mathrm{d}}			% Abk�rzung f�r Differential/Integral d
\newcommand{\del}{\partial}				% Partielle Ableitung
\newcommand{\bal}{\begin{aligned}}
\newcommand{\eal}{\end{aligned}}
\newcommand*\widefbox[1]{\fbox{\hspace{2em}#1\hspace{2em}}}
\renewcommand\citeform[1]{#1}


\setcounter{secnumdepth}{3}
%\setlength{\subfigtopskip}{cm}
% ==================  New Chapter/Section Definition  =====================

%\titleformat{?command?}[?shape?]{?format?}{?label?}{?sep?}{?before-code?}[?after-code?]
\titlespacing{\chapter}{0pt}{0pt}{20pt}
\titleformat{\chapter}[display]{\bfseries\Large \color{MyBlue}}{\filleft \fontsize{50}{20} \selectfont \thechapter}{.5em}{\titlerule \vspace{2ex}}[\vspace{2ex} \titlerule]
\titleformat{\section}
  {\normalfont \filcenter \large\bfseries \color{MyBlue}}{\thesection}{1em}{}
%========================= Document ==============================
% =============================Titelseite ==================================

\begin{document}

\pagenumbering{roman}

\begin{titlepage}
\vspace{-2cm}
\begin{center}
\textbf{\LARGE Master - Arbeit }
\vspace{1cm}
% Title
\hrule height 2pt
\vspace{1.0cm}
{\huge  \bfseries Title of Master Thesis}\\
\vspace{1.0cm}
\hrule height 2pt
\vspace{1cm}



% Upper part of the page

\textbf{vorgelegt von}\\
\vspace{1cm}
\textbf{\Large David Symhoven}\\
\vspace{1cm}
\textbf{an der}\\

\begin{figure}[H]
	\centering
		\includegraphics[width=0.30\textwidth]{lmusiegel} %Bild 1
\end{figure}

%\includegraphics[width=0.3\textwidth]{logo.jpg}\\
\textsc{\large Fachbereich Physik}\\
\textsc{\large Lehrstuhl f�r Plasma and Computational Physics}\\
\vspace{1cm} 

\textbf{Gutachter:}\\
\vspace{0.5cm}
\textsc{Prof.Dr.Hartmut Ruhl}\\[0.5cm]


\vspace{1.5cm}

\textbf{M�nchen, 2017}

\vfill


% Bottom of the page


\end{center}

\end{titlepage}


%================================= ERKL�RUNG=============================
\chapter*{Declaration}
\addcontentsline{toc}{chapter}{Declaration}
\vspace{1cm}
\noindent
Erkl�rung:\\
\newline
Hiermit erkl�re ich, die vorliegende Arbeit selbst�ndig verfasst zu haben und keine anderen als die in der Arbeit angegebenen Quellen und Hilfsmittel benutzt zu haben.\\
\newline
\vspace{2cm}
\noindent
M�nchen, Datum der Abgabe

\vspace{2cm}

\line(1,0){300}\\
M�nchen, 18.07.2017, David Symhoven


%================================= ABSTRACT==============================
\begin{abstract}
\begin{center}
\textbf{Abstract:}\\
Blah Blah Blah Mr. Freeman
\end{center} 
\end{abstract}

%================================= SYMBOLE UND KONTANTEN ================
\chapter*{Symbols and Constants}


\begin{tabular}{lll}
%Plank'sches Wirkungsquantum  	& \qquad h  			&  \qquad 6.62606957(29) $\cdot 10^{-34}~\si{\joule\second}$ \\ 
%Plank'sches Wirkungsquantum	  	&  \qquad $\hbar$  		&  \qquad1.054571726(47) $\cdot 10^{-34}~\si{\joule\second}$ \\
%Boltzmann - Konstante			& \qquad $k_{B}$ 		& \qquad  1.3806488(13) $\cdot 10^{-23}~\si{\joule\per\kelvin}$\\
%Avogadro - Konstante			& \qquad $N_A$ 		& \qquad 6.02214129(27) $\cdot 10 ^{23}~\si{\per\mole}$\\
Vacuum permittivity				& \qquad $\epsilon_0$	& \qquad 8.85418781762 $\cdot 10 ^{-12}~\si{\ampere\second\per\volt\per\meter}$\\
Vacuum permeability				& \qquad $\mu_0$		& \qquad 2566370614 $\cdot 10 ^{-6}~\si{\newton\per\square\ampere}$\\
%atomare Masseneinheit			& \qquad u 			& \qquad 1.660538921(73) $\cdot 10^{-27} ~\si{\kilogram}$\\
%Elektronenvolt					& \qquad eV			& \qquad 1.602176565(35) $\cdot 10^{-19}~\si{\joule}$\\
\noalign{\vskip 5mm}
Electrical flux density				& \qquad $\vec{D}$		& \qquad $[\si{\ampere\second\per\square\meter}]$\\
Magnetic flux density				& \qquad $\vec{B}$		& \qquad $[\si{\tesla}]$\\
Magnetic field strength			& \qquad $\vec{H}$		& \qquad $[\si{\ampere\per\meter}]$\\
Electric field stength				& \qquad $\vec{E}$		& \qquad $[\si{\volt\per\meter}]$\\
%1 Angstr�m 					& \qquad $\si{\angstrom}$  	& \qquad  $10^{-10}~\si{\meter}$ \\
%1 Nanosekunde 				& \qquad $\si{\nano\second}$  	& \qquad $10^{-9}~\si{\second}$ \\
%1 Pikosekunde 					& \qquad $\si{\pico\second}$ 	& \qquad $10^{-12}~\si{\second}$ \\
%1 Femtosekunde				& \qquad $\si{\femto\second}$ 	& \qquad $10^{-15}~\si{\second}$ \\

\noalign{\vskip 5mm}

%Ort							& \qquad $\vec{r}$			& \qquad $[\si{\meter}]$\\
%Geschwindigkeit				& \qquad $\vec{v}$ 			& \qquad $[\si{\meter\per\second}]$ \\
%Beschleunigung				& \qquad $\vec{a}$ 			& \qquad $[\si{\meter\per\square\second}]$ \\ 
%Impuls						& \qquad $\vec{p}$			& \qquad $[\si{\kilogram\meter\per\second}]$ \\
%Kraft							& \qquad $\vec{F}$ 			& \qquad $[\si{\newton}]$ \\
%Masse						& \qquad m 				& \qquad $[\si{\kilogram}]$ \\
%Energie						& \qquad E 				& \qquad $[\si{\joule}]$ \\
%Temperatur					& \qquad T				& \qquad $[\si{\kelvin}]$ \\
%Druck						& \qquad p 				& \qquad $[\si{\newton\per\square\meter}]$ \\	
%Entropie						& \qquad S				& \qquad $[\si{\joule\per\kelvin}]$ \\
%Potential						& \qquad V				& \qquad \text{nicht eindeutig} \\
%chemisches Potential			& \qquad $\mu$			& \qquad \text{nicht eindeutig} \\
%Zeit 							& \qquad t 				& \qquad $[\si{\second}]$\\
%diskretisierte Zeit				& \qquad $\Delta t$			& \qquad $[\si{\second}]$\\
%Frequenz						& \qquad $\omega$			& \qquad $[\si{\per\second}]$\\
%Gesamtteilchenanzahl			& \qquad N				& \\
%Anzahl der Freiheitsgerade		& \qquad $f$				& \\
%\noalign{\vskip 5mm}

Nabla - Operator				& \qquad $\nabla$			& \qquad  $\left( \frac{\del }{\del r_1}, \ldots, \frac{\del }{\del r_n}\right)$ \\
\noalign{\vskip 2mm}
Laplace - Operator				& \qquad $\Delta$			& \qquad  $\sum_{i=1}^{n} \frac{\del^2}{\del r_i^2}$\\
%\noalign{\vskip 2mm}
%Hamilton - Operator				& \qquad $\mathcal{H}$		& \qquad $\mathcal{H} = -\frac{\hbar^2}{2m}\Delta + V(\vec{r})$ \\		
%\noalign{\vskip 2mm}
%Lagrange - Funktion				& \qquad $\mathcal{L}$		& \qquad $\mathcal{L} = T - V$ \\

\end{tabular}


%================================= Inhaltsverzeichnis ==========================
\tableofcontents

%============================ Beginn Textseiten ================================
\pagenumbering{arabic}



%================================= Bildvorlage ====================================

%Code f�r Bildumgebung
%\begin{figure}[H]
	%\centering
	%	\includegraphics[width=0.50\textwidth]{Destruktive_Interferenz.jpg} %Bild 1
	%\caption[Destruktive Interferenz]{Destruktive Interferenz \cite{[Wiki12]}}
	%\label{fig:Interferenzfilterr}
%\end{figure}

%===========================================================================


%======================== EINLEITUNG ========================
%===========================================================
\chapter{Einleitung}
Now it's going loose ...

\chapter{Grundlagen}
\section{Li�nard-Wiechert Potentiale}
\section{Numerik}
\subsection{Bewegungsgleichung}
\subsection{Euler-Verfahren}
\subsection{Leap-Frog-Verfahren}
\subsection{Boris-Pusher}
\subsection{Vay-Pusher}
\section{Hybride Felder}
\subsection{Maxwell-Gleichungen}
\subsection{Maxwell-Solver}
\subsection{Nah-und Fernfelder}





\end{document}











